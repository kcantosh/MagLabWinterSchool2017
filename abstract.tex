\documentclass[12pt]{article}
\usepackage{helvet}
\usepackage[margin=1in]{geometry}

\usepackage{fancyhdr}
\pagestyle{fancy}


\begin{document}
\title{Correlated electron systems using quantum Monte Carlo: a tutorial}
\author{Lucas K. Wagner}
\maketitle

\section*{Introduction}

In this tutorial, we will study H$_2$ very carefully using quantum Monte Carlo techniques. 
While at first blush this may seem like a very simple system (only two electrons), we will discover that it has many of the features that more complex systems also have, while a numerically exact solution can be found in a few seconds using quantum Monte Carlo.
We will thus be able to perform many calculations which will allow us to examine the physics of this system systematically in a short time. 

\section*{Learning goals} 
\begin{itemize}
\item How to perform a quantum Monte Carlo calculation on a first principles model of a chemical system
\item The effect of electron correlation on the ground state and excited states of a system.
\item The different types of correlation: short range and long range.
\item The relationship between the first principles calculation and effective models of the behavior.
\end{itemize}

\section*{Procedure} 
\begin{itemize}
\item Run QMC calculations on H$_2$ across different atomic separations.
\item Try different correlated wave functions
\item Examine how the effective model that describes H$_2$ changes with atom separation and with the treatment of short range correlation. 	
\end{itemize}

\paragraph{H$_2$ at different atomic separations.} 
At very large separations, the system is basically two weakly interacting atoms, while at small separations, the system is much better described by both electrons occupying a bonding orbital.
Because we are using exact quantum calculations, we can also examine the in-between case, when the atoms are at moderate separation.

\paragraph{Different many-body wave functions.} 
We consider three types of wave functions: no short range correlation (two independent Slater determinants), parameterized dynamic (short-range) correlation (multiple Slater-Jastrow wave function), and projection Monte Carlo solutions (exact in this case).
We see that sometimes the parameterized correlation is good, and sometimes it can fail to perfectly describe the correlation. 
The dynamic correlation affects the energy dramatically because the energy scale when two electrons are close is large.

\paragraph{The effective model.}
Often in physics we are interested only in the low-energy excitations of a material system. 
We learn about how the effective model is changed by how well we calculate the properties of the material system. 
In particular, the effective interactions are actually reduced by good treatment of electron correlation. 


\paragraph{Similarities and differences from more complex materials.}
The main difference is that for more complex materials there are usually more electrons around--and more options. 
The short range correlation is always quite important in multi-electron atoms, while in H$_2$, its importance decreases as the atoms become more separated.
This means that the importance of screening is much larger in multi-electron atoms, so methods that treat it well.
Many of the conversations in condensed matter physics concerns whether to start from the atomic limit and view the interatomic hopping as a perturbation, or to start from the delocalized picture and add interactions in. 
In this exercise, we interpolate between those two limits and see where neither picture is quite right.

\end{document}
