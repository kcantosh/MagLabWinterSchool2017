\documentclass[12pt]{article}
\usepackage{helvet}
\usepackage{fullpage}
\usepackage{amsmath}

\begin{document}

\section*{Introduction}

We are going to do a simulation of H$_2$ in this exercise. 
While it may seem small, this simple system includes a lot of the physics that is present in more complex materials, while taking a few seconds\footnote{The techniques we're using are not the fastest for the specific problem of H$_2$. However, they scale very well, so you can apply exactly the same techniques to systems up to around 1000 electrons, with a big enough computer.} to perform an exact simulation on. 
This way we can try a lot of different calculations!

\section*{Library: }

\subsection*{Class: H2Runner}

We have made a small Python class that simplifies the setup of a QMC run. You use it like the following:

\begin{verbatim}
from runqmc import H2Runner
import pickle

basename="test"
runner=H2Runner()
runner.r=1.4
runner.wavefunction="singlet"
runner.run_all(basename)
with open(basename+".pickle",'wb') as f:
  pickle.dump(runner,f)
\end{verbatim}

Try entering that into tmp.py and running it like
\begin{verbatim}
python3 tmp.py
\end{verbatim}
It will automatically run 3 calculations: no Jastrow factor, multiple Slater-Jastrow wave function, and diffusion Monte Carlo. 
All results will be stored in the file ``test.pickle'' for later analysis.

While you are reading through this document, go ahead and run 
\begin{verbatim}
python3 scan.py 	
\end{verbatim}
which will generate a scan over different bond lengths and spin states.

\begin{table}
\caption{Options and files in the H$_2$ helper scripts.}
\begin{tabular}{lp{0.5\columnwidth}}
File & Function \\
\hline
runqmc.py & Interface with QWalk\\
\hline
scan.py & Run QMC calculations \\
\hline
plot*.py & Plot results \\
\hline
\end{tabular}
\vspace{1cm}

\begin{tabular}{lp{0.3\columnwidth}p{0.5\columnwidth}}
Option & Meaning & Allowed values \\
\hline
wavefunction & Wave function (spin state) &  'singlet','triplet'\\
\hline
optimize\_det & Whether to optimize the determinant coefficients in Eqn~\ref{eqn:singlet_twf} &True/False \\
\hline
r & Bond length & real number greater than zero \\
\hline
\end{tabular}
\end{table}


\subsection*{Hamiltonian: H2Runner.gen\_sys() }

\subsection*{Trial wave function: Slater part}

To construct our wave function, note that there are two sites, let's call them $|1\rangle$ and $|2\rangle$. 
These represent the 1s atomic orbitals around atoms 1 and 2, respectively. 
Since there are two electrons, we can represent the two-particle quantum states (with one up and one down electron) as: \\
\begin{tabular}{lc}
$|11\rangle$ & Both on atom 1 \\
$|12\rangle$ & Up on atom 1, down on atom 2 \\
$|21\rangle$ & Up on atom 2, down on atom 1 \\
$|22\rangle$ & Both on atom 1 \\
\end{tabular}\\

For the {\bf singlet} state($s=0$),\footnote{In this notation, we are always implicitly multiplying the real-space wave function by the spin state. That's going to be $|\uparrow\downarrow \rangle - |\downarrow\uparrow\rangle$ for the singlet state, and $|\uparrow\uparrow\rangle$ for the triplet state} the real-space part of the wave function must be symmetric. 
We can thus parameterize the wave function as 
\begin{equation}
|\Psi_S\rangle = \frac{a}{\sqrt{2}}\left(|12\rangle + |21\rangle\right) + \frac{b}{\sqrt{2}}\left(|11\rangle + |22\rangle\right),
\label{eqn:singlet_twf}
\end{equation}
where $a^2+b^2=1$. 
If you prefer, this wave function can be written in real space:
\begin{equation}
\left[a(\chi_1(r_1)\chi_2(r_2)+\chi_2(r_1)\chi_1(r_2)) + b(\chi_1(r_1)\chi_1(r_2)+\chi_2(r_1)\chi_2(r_2)).\right]	
\end{equation}


There are two interesting limits for Eqn~\ref{eqn:singlet_twf}. 
The non-interacting limit is when $a=b$. 
In this case, you can verify that our wave function is equivalent to a product between single particle bonding orbitals $|b\rangle=|1\rangle + |2\rangle$.
This is the Hartree-Fock solution.
Another interesting limit is when $a=1,b=0$. 
In that case there is no double occupancy. 
This is the `antiferromagnetic' orientation, in which one atom always has an up electron when the other has a down, and vice-versa.


For the {\bf triplet} $s=1$ case, the wave function in this basis has no parameters:
\begin{equation}
|\Psi_T\rangle = \frac{1}{\sqrt{2}}\left(|12\rangle - |21\rangle\right).
\label{eqn:triplet_twf}
\end{equation}
Sometimes, you will hear that high-spin states are more weakly correlated. 
This is a very simple example of why that is often true; there are just fewer states available when the electrons are the same spin.

You can select between triplet and singlet by setting H2Runner.wavefunction.

\subsection*{Trial wave function: Jastrow part}

Our wave functions in Eqns~\ref{eqn:singlet_twf} and~\ref{eqn:triplet_twf} have a serious deficiency for realistic chemical systems. 
They do not include local correlation within the atom. 
At ranges less than the bond length, the electrons are completely uncorrelated, which means that an up and a down electron do not avoid each other at that scale.

To describe the short-range effects, we add a {\bf Jastrow factor}.
It's perhaps easiest to see this in real space.
For the two electrons in H$_2$, the wave function will be:
\begin{align}
\Psi_{SJ}(r_1,r_2)=&\left[a(\chi_1(r_1)\chi_2(r_2)+\chi_2(r_1)\chi_1(r_2)) + b(\chi_1(r_1)\chi_1(r_2)+\chi_2(r_1)\chi_2(r_2)) \right] \notag \\
\times& \exp\left[f_{1b}(r_1)+f_{1b}(r_2) + f_{2b}(r_{12})\right],
\label{eqn:slater-jastrow}	
\end{align}
where $r_{12}=|r_1-r_2|$. $f_{2b}$ is the two-particle part, which introduces explicit dependence on the particle-particle distance into the wave function.
It includes the cusp condition when two electrons are at the same place. 
This factor tends to push electrons apart from one another, which can be fixed by the one-body part $f_{1b}$. 
We expand $f_{1b}$ and $f_{2b}$ in a basis that decreases with distance and optimize the coefficients. 


\section*{Things to do.}

\begin{itemize}
\item Look at total energy for different wave functions. What happens to the singlet-triplet energy difference as a function of $r$?
\item Energy difference between singlet and triplet as a function of $r$.
\item How does the double occupancy change as we increase $r$?
\item How does including Jastrow correlation change the double occupancy? 
\item Radial distribution function. What changes in triplet versus singlet? What changes between the different wave functions?
\end{itemize}

\section*{Contrasting with the Hubbard model}

The wave function {\it ansatz} in Ean~\ref{eqn:singlet_twf} is exact for the 2-site Hubbard model. 
The eigenvalue equation for $a$ and $b$ is 
\begin{equation}
	\begin{bmatrix}
	  0 & -2t \\
	  -2t & U \\
    \end{bmatrix} 
\begin{bmatrix}
         a \\
         b
        \end{bmatrix} 
    =
E\begin{bmatrix}
         a \\
         b
        \end{bmatrix} 
\end{equation}

While we can solve the two-site Hubbard system analytically fairly easily, we will solve this numerically in this case. 
From the {\it ab-initio} results, we can determine the effective $t$ and $U$ by matching 


%the characteristic equation  is 
%\begin{equation}
%E^2-EU-4t^2=0.
%\end{equation}
%with 
%\begin{equation}
%E=\frac{U\pm \sqrt{U^2+16t^2}}{2}	
%\end{equation}

We can determine $t$ and $U$ if we have two 

\begin{itemize}
\item Singlet-triplet gap
\item Double occupancy
\item Is the Hubbard model a good model for the physics of this system? If we don't have enough information, how can we determine this?
\end{itemize}


\end{document}
