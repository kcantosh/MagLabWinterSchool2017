\documentclass[12pt]{article}
\usepackage{helvet}
\usepackage{fullpage}

\begin{document}

\section*{Introduction}

We are going to do a simulation of H$_2$ in this exercise. 
While it may seem small, this simple system includes a lot of the physics that is present in more complex materials, while taking a few seconds\footnote{The techniques we're using are not the fastest for the specific problem of H$_2$. However, they scale very well, so you can apply exactly the same techniques to systems up to around 1000 electrons, with a big enough computer.} to perform an exact simulation on. 
This way we can try a lot of different calculations!

\section*{Library: }

\subsection*{Class: H2Runner}

We have made a small Python class that simplifies the setup of a QMC run. 
It writes a few simple 

\begin{table}
\begin{tabular}{lp{0.5\columnwidth}}
Option & Allowed values \\
\hline
wavefunction & 'singlet','triplet'\\
optimize\_det & True/False \\
r & real number greater than zero \\
\end{tabular}
\end{table}


\begin{table}
\begin{tabular}{lp{0.5\columnwidth}}
File & Function \\
\hline
runqmc.py & Interface with QWalk\\
test.py & Run QMC calculations \\
plot*.py & Plot results \\
\end{tabular}
\end{table}


\subsection*{Hamiltonian: H2Runner.gen\_sys() }

\subsection*{Trial wave function: Slater part}

To construct our wave function, note that there are two sites, let's call them $|1\rangle$ and $|2\rangle$. 
These represent the 1s atomic orbitals around atoms 1 and 2, respectively. 
Since there are two electrons, we can represent the two-particle quantum states (with one up and one down electron) as: \\
\begin{tabular}{lc}
$|11\rangle$ & Both on atom 1 \\
$|12\rangle$ & Up on atom 1, down on atom 2 \\
$|21\rangle$ & Up on atom 2, down on atom 1 \\
$|22\rangle$ & Both on atom 1 \\
\end{tabular}\\

For the {\bf singlet} state($s=0$),\footnote{In this notation, we are always implicitly multiplying the real-space wave function by the spin state. That's going to be $|\uparrow\downarrow \rangle - |\downarrow\uparrow\rangle$ for the singlet state, and $|\uparrow\uparrow\rangle$ for the triplet state} the real-space part of the wave function must be symmetric. 
We can thus parameterize the wave function as 
\begin{equation}
|\Psi_S\rangle = a\left(|12\rangle + |21\rangle\right) + b\left(|11\rangle + |22\rangle\right),
\label{eqn:singlet_twf}
\end{equation}
where $a^2+b^2=1$. 
There are two interesting limits for Eqn~\ref{eqn:singlet_twf}. 
The non-interacting limit is when $a=b$. 
In this case, you can verify that our wave function is equivalent to a product between single particle bonding orbitals $|b\rangle=|1\rangle + |2\rangle$.
This is the Hartree-Fock solution.
Another interesting limit is when $a=1/\sqrt{2},b=0$. 
In that case there is no double occupancy. 
This is the `antiferromagnetic' orientation, in which one atom always has an up electron when the other has a down, and vice-versa.


For the {\bf triplet} $s=1$ case, the wave function in this basis has no parameters:
\begin{equation}
|\Psi_T\rangle = \frac{1}{\sqrt{2}}\left(|12\rangle - |21\rangle\right).
\label{eqn:triplet_twf}
\end{equation}
Sometimes, you will hear that high-spin states are more weakly correlated. 
This is a very simple example of why that is often true; there are just fewer states available when the electrons are the same spin.

You can select between triplet and singlet by setting H2Runner.wavefunction.

\subsection*{Trial wave function: Jastrow part}

Our wave functions in Eqns~\ref{eqn:singlet_twf} and~\ref{eqn:triplet_twf} have a serious deficiency for realistic chemical systems. 
They do not include local correlation within the atom. 
At ranges less than the bond length, the electrons are completely uncorrelated, which means that an up and a down electron do not avoid each other at that scale.

To describe the short-range effects, we add a {\bf Jastrow factor}.
For 

\section*{Contrasting with the Hubbard model}

A common approximation for H$_2$ is the two-site Hubbard model. 
In this model, our 

\begin{itemize}
\item 	
\end{itemize}


\end{document}
